% introduction for tunix.tex

\chapter{Introduction}
%%%%%%%%%%%%%%%%%%%%%%%%%
% The idea behind TUn!x %
%%%%%%%%%%%%%%%%%%%%%%%%%
\section{The idea behind \tunix}
The original idea for \tunix was born out of the problem that computer science
students in their first semester had to set up their computers with a special
version of the Java software together with a specialized library for their
first-semester introductory course in programmming. Since quite a few students
who enroll have never seriously used a computer before, this proved difficult
for many and therefore represented an additional barrier for becoming a computer
science student (and graduate).

Additionally, it was observed that many students never use a Unix-style
environment out of fear that it is too hard to set up and in-depth knowledge of
the command-line is necessary, even for everyday tasks. Sometimes students go to
great lengths to avoid using Linux, even to the point were students choose their
branch of study based on the requirement for the use of Linux. The thought that
some students graduate with a bachelors degree without ever having used anything
but Microsoft Windows is troublesome, to say the least.\\

Thus, the idea for \tunix was born. \tunix is a bootable and installable Linux
live-CD based on Ubuntu that allows you to boot your PC with a full Linux system
that doesn't touch your harddisk(s), eliminating the risk of data-loss. The
designated goal of \tunix is to provide a ready-to-use Linux environment that
requires the least-possible knowledge of Linux to run, taking into account that
it would be run exclusively by first-semester students of the Technical
University of Vienna.

The \tunix live-CD includes all the software students could need in their first
semesters. The software-versions are coordinated with the environment used in the
courses, to keep problems with differing software-versions at a minimum.
Installed software is customized so it has more useful defaults, making it
easier to use.

The first release of \tunix was made in time for the wintersemester 2005/06 and
was based on Ubuntu 5.04. The second release, documented here, was released in
time for the wintersemester 2006/07 and is based on Ubuntu 6.06.1. It brings
many improvements over the first version and is the first version where the
\tunix CD is actually installable (the previous version didn't come with an
installer). See chapter \ref{changes-overview} on page \pageref{changes-overview}
for what has changed compared to Ubuntu for this release.

%%%%%%%%%%%%%%%%%%%%%%%%%%%%%%%%%%%%
% TUn!x is based on other projects %
%%%%%%%%%%%%%%%%%%%%%%%%%%%%%%%%%%%%
\section{\tunix is based on other projects}
As with most free software projects, \tunix would not be possible without the
work of many other free software projects. This chapter is dedicated to
mentioning these projects.\\

\tunix itself is based entirely on Ubuntu\footnote{\href{http://www.ubuntu.com}%
{http://www.ubuntu.com}}, which is in turn mostly based on the Debian\footnote{%
\href{http://www.debian.org}{http://www.debian.org}} Linux distribution. With
that lineage comes the use of the DPKG and APT packet management systems.

The web presence of \tunix is based on MediaWiki\footnote{\href{%
http://www.mediawiki.org}{http://www.mediawiki.org}}, the free Wiki-software
originally developed for Wikipedia\footnote{\href{http://www.wikipedia.org}%
{http://www.wikipedia.org}}. Several 3\textsuperscript{rd}-party extensions are installed on
the \tunix homepage, for a complete list please refer to
Spezial:Version\footnote{\href{http://pluto.htu.tuwien.ac.at/Spezial:Version}{%
http://pluto.htu.tuwien.ac.at/Spezial:Version}}.

Software used for developing \tunix includes Subversion\footnote{\href{%
http://subversion.tigris.org}{http://subversion.tigris.org}},
VMware\footnote{\href{http://www.vmware.com}{http://www.vmware.com}} and of
course the package building tools for Debian based distributions.

%%%%%%%%%%%%%%%%%%%%%%%%%%%%%%
% A list of the changes made %
%%%%%%%%%%%%%%%%%%%%%%%%%%%%%%
\section{A list of the changes made}\label{changes-overview}
\tunix features many improvements over Ubuntu, a short overview is given here:
\begin{description}
   \item[Installed software:] The following software-packages are preinstalled in \tunix
   but not in Ubuntu:
     \begin{itemize}
       \item \textbf{sun-java5-sdk:} The official Sun Java development kit as used in
         the courses "Einführung ins Programmieren"\footnote{Introduction to %
	 programming} and "Algorithmen und Datenstrukturen"\footnote{Algorithms
	 and data structures}.
       \item \textbf{postgresql-8.1:} PostgreSQL is used in "Datenmodellierung"
         \footnote{Data modeling}.
       \item \textbf{dia:} Dia is also used in "Datenmodellierung" 
       \item \textbf{vpnc:} This Cisco-compatible VPN client can be used to make a
         secure Virtual Private Network connection to the university.
       \item \textbf{gajim:} A Jabber-client that has more features than
         gaim\footnote{Gaim is the previous name of the software now known as
         pidgin}.
       \item \textbf{sysv-rc-conf:} A simple ncurses-based tool to easily
         configure your init-scripts.
     \end{itemize}
   \item[Removed software:] The following software was removed from the \tunix
   live-CD:
     \begin{itemize}
       \item \textbf{example-content:} Only contains a few videos and eats up a
         lot of space.
       \item \textbf{gaim:} Replaced by gajim (see above).
       \item \textbf{thunderbird-locale-en-gb} and \textbf{openoffice.org-l10n-en-za}
         and the meta-packages depending on them, \textbf{language-support-en} and
	 \textbf{ubuntu-live}.
       \item \textbf{Windows software:} The Ubuntu live-CD contains some free Windows
         software that was removed on the \tunix version.
     \end{itemize}
   \item[Updated software:] The entire software was updated to the most recent
     version (as of 2006-09-12) found in the Ubuntu repositories. 
   \item[Configuration changes:] Several packages had their default
     configuration changed (see chapter \ref{chapter:modifications}, p. 
     	\pageref{chapter:modifications} for a detailed description of how to make
	these changes):
     \begin{itemize}
       \item \textbf{apt:} The Apt package management system also downloads from
         the \tunix apt-repository. This allows us to issue updates to software
	 and configuration coming from us.
       \item \textbf{gnome/gtk/gdm:} The Gnome desktop environment, the Gnome
         display manager and the GTK+ graphics toolkit all feature a unique
	 theme.
       \item \textbf{firefox:} The Firefox webbrowser has customized bookmarks
         linking to the essential websites a student should know about.
       \item \textbf{screen:} GNU Screen uses UTF8 per default and most
         importantly has a nice taskbar at the bottom.
       \item \textbf{vpnc:} Example configuration files for use with the WLAN at
         the university and from outside of the university are available in the
	 \file{/etc}-directory.
       \item \textbf{usplash:} Usplash is responsible for the splash-screen
         shown during boot and displays the \tunix-tux instead of the Ubuntu
	 logo.
     \end{itemize}
\end{description}
