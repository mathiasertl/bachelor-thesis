% frontmatter for tunix.tex

\maketitle

%%%%%%%%%%%%%%%
%%% Preface %%%
%%%%%%%%%%%%%%%
\chapter{Preface}
\tunix{} is a Linux distribution customized to fit the needs of students who study
computer science at the Technical University of Vienna. \tunix{} is based on
Ubuntu, a popular Linux distribution designed for the desktop, which is in turn
based on the Debian Project.

\tunix{} has a yearly release schedule and is developed each year by two different 
computer science students together with AO.Univ.Prof.\ Dr.\ Peter Purgathofer. 
This is the documentation on what has been done and how in the second
edition of \tunix, released in September, 2006. In this year, \tunix{} was developed
by Mathias Ertl and Florian Cech. This document is split into 4 parts:
\begin{description}
\item[Chapter 1] gives you an idea of what \tunix{} is.
\item[Chapter 2] is a collection of ideas and concepts of what could be (and only
partly was) done with \tunix.
\item[Chapter 3] explains the process of remastering an Ubuntu live-CD.
\item[Chapter 4] explains DPKG-packages and APT-repositories
\item[Chapter 5] describes the actual changes done to \tunix.
\end{description}

%%%%%%%%%%%%%%%%%%
%%% Thank You! %%%
%%%%%%%%%%%%%%%%%%
%\chapter{Thank You!}
%Peter Purgathofer, Faculty of Computer Science, Fachschaft Informatik, Mux,
%Ekimus\todo{write a thank you chapter}.

\tableofcontents
\lstlistoflistings
%\listoffigures
%\listoftables
