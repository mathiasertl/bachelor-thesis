% package-management for tunix.tex

%%%%%%%%%%%%%%%%%%%%%%%%%%
%%% Package Management %%%
%%%%%%%%%%%%%%%%%%%%%%%%%%
\chapter{Package management}
Debian, and thus Ubuntu, and thus \tunix is based on two package management
systems working together:
\begin{description}
  \item[dpkg,] or Debian Package Management System, is used to locally install
    packages called \emph{.deb packages} (see section \ref{chapter:deb packages%
    }, p. \pageref{chapter:deb packages}). It is unable to resolve dependencies
    or get new packages from the internet and can only handle files
    available locally. The enduser rarely interacts with dpkg directly.
  \item[apt,] or \textbf{A}dvanced \textbf{P}ackaging \textbf{T}ool, is used to
    complement dpkg. Apt is able to interact with central (possibly remote)
    sources for new and updated software, called \emph{apt-repositories} (see
    section \ref{chapter:apt repositories}, p. \pageref{chapter:apt %
    repositories}) and is able to automatically resolve dependencies.
    
    Apt itself never installs the software itself, but uses a
    dpkg as a backend\footnote{Note that while the most common backend is dpkg,
    there are also different backends available, most notably for the Redhat
    Package Management system.} to install software. Many more experienced
    users use apt directly to install software but there are also many
    easy-to-use graphical frontends available.
\end{description}
This software team is at the heart of every Debian based Linux distribution,
in fact, the term "Debian based" is normally used to describe "based on
dpkg/apt"\footnote{One could distinguish between "Debian based" and "dpkg/apt
based", though. A dpkg/apt based distribution doesn't necessarily include
packages coming from the Debian project, but Debian based distributions (hence
the name) do. Ubuntu and thus \tunix both include a great number of debian
packages.}.

%%%%%%%%%%%%%%%%%%%%%
%%% .deb packages %%%
%%%%%%%%%%%%%%%%%%%%%
\section{.dep packages}\label{chapter:deb packages}
.deb Packages are single files (file suffix is .deb) containing a software
package. Binary packages are used by dpkg, the Debian Package Manager, to
install new software, source packages are not directly installed but can be
fetched from apt-repositories using \cmd{apt-get source \$package}. 
There are three kinds of packages:
\begin{description}
  \item[source packages] contain the original source in human-readable form
    together with a little meta information. Source packages frequently come
    with additional patches. These packages are used to generate binary
    packages and also enable the user to compile the software themselves (e.g.
    with different compiler flags).
  \item[binary packages] usually contain software in a ready-to-use (e.g.
    compiled) form. These are the packages usually installed (indirectly,
    through apt) by the user. Note that often multiple binary packages come from
    one source package.
  \item[meta packages] are specialized binary packages that contain no software
    but only dependency information. They are useful for distributing a set of
    software. The most prominent example of such a package is the package
    \pkg{ubuntu-desktop} that depends on all software that comes with Ubuntu.
\end{description}

This section explains how .deb packages are structured, how to build new
binary packages either from scratch or by modifying existing ones. \emph{But}
one important aspect of real-world package maintenance is left out: We don't
explain how binary packages are automatically generated from source packages or
how source packages in general work. This process was not required for \tunix,
since it contains no packages that have to be compiled into a binary
form\footnote{Almost: The \tunix Usplash image is actually a PNG image file that
is compiled to a shared object, but we saw no use in creating a source package
together with an additional repository when simply publishing the original PNG
would do the same job.}. Therefore, the following subsections refer to binary
packages exclusivly except where explicitly mentioned.

Official and more complete documentation on how .deb packages work is available 
in the Debian Policy Manual\cite{debian policy manual} and the Debian New
Maintainers Guide\cite{debian new maintainers guide}.

%%%%%%%%%%%%%%%%%%%%%%%
%%% Basic structure %%%
%%%%%%%%%%%%%%%%%%%%%%%
\subsection{Basic structure}
.deb packages are a relatively simple \emph{ar archive} that can be unpacked using the
tool with the same name. The process is explained in \lstlistingname{} \ref{code:%
deb-unpack} (p. \pageref{code:deb-unpack}).
\begin{lstlisting}[label={code:deb-unpack},caption={Unpacking a .deb package using
ar},float=ht]
$\sim$/package $\$$ ar xf tunix_1.0.0-1_i386.deb
$\sim$/package $\$$ ls
data.tar.gz debian.tar.gz debian-binary
\end{lstlisting}

The file name of the package \emph{must} follow the following convention:
\begin{quote}
    \$softwarename\_\$version\_\$arch.deb
\end{quote}
in which \$softwarename is the name of the package, \$version is the version of
the software (the part after the first '-' represents the version of the
package itself) and \$arch the architecture the package is for. (The special
string 'all' is used for packages that work on all supported plattforms.)

As shown in \lstlistingname{} \ref{code:deb-unpack} (p. \pageref{code:%
deb-unpack}) a .deb package contains \emph{exactly} three files:
\begin{description}
  \item[\file{data.tar.gz}] contains the files as they will be extracted onto the
    filesystem hierarchy. The contents of this archive depend mainly on the
    original software but must also contain some documentation from the package
    maintainer. The requirements are described in section \ref{subsection:%
    installation files} (p. \pageref{subsection:installation files}).
  \item[\file{debian.tar.gz}] contains all meta-information and may also contain
    scripts to aid the installation and removal of the software. The most common
    content of this archive is described in section \ref{subsection:meta
    information} (p. \pageref{subsection:meta information}).
  \item[\file{debian-binary}] contains the version of the debian standards this
    package adheres to. This file is implicitly generated in the package building
    process (section \ref{subsection:package building}, p. \pageref{subsection:%
    package building}).
\end{description}

%%%%%%%%%%%%%%%%%%%%%%%%%%
%%% Installation files %%%
%%%%%%%%%%%%%%%%%%%%%%%%%%
\subsection{Installation files}\label{subsection:installation files}
The \file{data.tar.gz} archive has to contain only three installation files
(all located in \file{/usr/share/doc/\$package}) to make it compliant
with Debian standards: 
\begin{description}
  \item[copyright] is a freeform copyright notice that should contain the
    authors of the package and under what licence the package itself is.
  \item[changelog.gz] is the gz-compressed (maximum compression) upstream
    changelog as distributed by the author(s). Since this file doesn't come from
    the package maintainer, it is also freeformed.
  \item[changelog.Debian.gz] is also a gz-compressed changelog, but describes
    the changes made to the debian package. Every version of the package
    \emph{must} have an entry here. Since this file is used by frontends to
    apt/dpkg, it has to follow a strict syntax. The exact syntax is given in
    EBNF-form in \lstlistingname{} \ref{code:changelog-debian syntax} (p.
    \pageref{code:changelog-debian syntax}) and a real-world example is given in
    \lstlistingname{} \ref{code:changelog-debian} (p. \pageref{code:changelog-%
    debian}). 
\end{description}

\begin{lstlisting}[label=code:changelog-debian syntax, caption={Syntax of the
changelog.Debian file in EBNF notation}]
file      = entry { , entry } ;
entry     = header , { newline , } desc_line , 
	{ desc_line , } { newline , } signature ;
header    = ? package ?, " (" , version , ") " , 
	{ ? dist ? } , "; urgency=" , urgency , "\n" ;
newline   = "\n" ;
desc_line = "  " , ? freeformed text ? , "\n";
signature = " --" , ? name ? , " <", ? email ? , ">  " ,
	? RFC822 date ? , "\n";
version   = ? software version ? , "-" , 
	? package version ? ;
urgency   = "low" | "medium" | "high" | "emergency" | 
	"critical" ;
\end{lstlisting}

\begin{lstlisting}[label=code:changelog-debian,caption={Example of the
changelog.Debian file}]
tunix-artwork (1.0.0-1) tunix; urgency=low
  
  * something has changed
    this is a freeformed text, so the asterisks are
    only there for convinience.
  * even more change details
 
 
 -- Mathias Ertl <m@ab.at>  Tue, 15 Aug 2006 17:46:22 +0200
\end{lstlisting}
The date in any changelog entry has to be in RFC822 format, you can get the
current date in this format with the tool \cmd{822-date}. The email-adress
in \lstlistingname{} \ref{code:changelog-debian} is kept deliberatly short to
fit the available page-width, but should in reality match the entry in
\file{DEBIAN/control}. Take special care that there are actually \emph{two}
spaces between the email-adress and the date.

%%%%%%%%%%%%%%%%%%%%%%%%
%%% Meta information %%%
%%%%%%%%%%%%%%%%%%%%%%%%
\subsection{Meta information}\label{subsection:meta information}
Binary packages have to contain several files of meta information. In a finished
binary package these files end up in \file{control.tar.gz}, for the process
of building the package, the files should go into the directory 
\file{DEBIAN/} (as opposed to the directory \file{debian/} for source
packages). A complete list of files that can go into the
\file{DEBIAN/} directory is available at \cite{debian policy manual}, this
section describes the most important ones (and how to create them manually).

\subsubsection{control}
\label{control}
The file \file{DEBIAN/control} is actually the most important one, as it
contains all the information necessary for the package building process,
dependency resolution (with apt), etc.. It is a plain text file that contains a
series of data fiels. Each data field contains a field label, a colon (':') and
the value of that field. If a value spans multiple lines, new lines must begin
with a space or a tab. The following list lists the most important field names
as well as what they should contain:
\begin{description}
  \item[Package:] (mandatory) The name of the package
  \item[Version:] (mandatory) Version including package version
  \item[Architecture] (mandatory) Architecture the package is for. May also be
    the special string "all" to indicate a platform-independent package.
  \item[Maintainer:] (mandatory) The package maintainers name and email adress.
    The adress must be inside the usual brackets ("<foo@bar.com>").
  \item[Description:] (mandatory) This is a multi-line field, where the first
    line gives the single line synopsys and a multi-line extended description.
    Lines only containing a '.' (of course the leading whitespace still has to
    be present) are treated as paragraph delimiter.
  \item[Depends et al] If a package depends on other packages you can specify
    them using \emph{Depends} for packages required by this package to run,
    and/or \emph{Pre-Depends} if packages are required during the setup process
    of this package. It is possible to narrow down the version of the required
    packages in a regular expression style syntax. For example, it is possible,
    to specify that package A depends on package B, which must be at least of
    version 1.0. You do not have to declare dependencies to packages which are
    marked with the \emph{Essential} field. 
  \end{description}
The \lstlistingname{} \ref{code:control file} (p. \pageref{code:control file})
gives a working example.
\begin{lstlisting}[label=code:control file, caption=Example of a DEBIAN/control
file, float=ht]
Package: tunix
Version: 1.0.1-1
Architecture: all
Depends: tunix-artwork, tunix-software, tunix-keyring,
 tunix-foo (>= 1.0.0)
Suggests: tunix-extras
Installed-Size: 8
Section: base
Priority: optional
Maintainer: Mathias Ertl <mati@pluto.htu.tuwien.ac.at>
Description: single line description
 .
 Multi line description. This will be
 printed as a single paragraph
 .
 And this is the next paragraph.
\end{lstlisting}

\subsubsection{md5sums}
This file simply contains checksums for all installation files. Before you
finally create the package and all installation files are in their final state,
you have to recreate the md5sums so the checksums are correct. The
\lstlistingname{} \ref{code:md5sums} (p. \pageref{code:md5sums}) explains how to
recreate these sums "by hand".
\begin{lstlisting}[caption=Recreating md5sums for DEBIAN/md5sums,
label=code:md5sums, float=ht]
$\sim$/work/tunix-1.0.1/ $\$$ ls
DEBIAN  usr
$\sim$/work/tunix-1.0.1/ $\$$ rm -f DEBIAN/md5sums
$\sim$/work/tunix-1.0.1/ $\$$ find * -type f | grep --invert-match\ 
	"^DEBIAN[$\$$/]" | xargs md5sum > DEBIAN/md5sums
\end{lstlisting}

\subsubsection{Maintainer scripts}
A package might require additional steps in the (un)installation process besides
unpacking the files found in \file{data.tar.gz} and installing all required
packages. For example, most daemons (i.e. apache) normally do not
run as root but have their own system user (called \emph{www-data} for apache)
that has to be added during the installation process and removed during the
uninstallation process. Some packages (i.e. java) also prompt for the agreement
to a license. More specialized code might be necessary if a package has to be
updated by apt, for example, if the syntax of the configuration file changes.
You can write up to four scripts for this purpose with the self-describing names
\cmd{preinst}, \cmd{postinst}, \cmd{prerm} and \cmd{postrm}. These scripts can
be an executable in binary format or a script written in any language. They do
have to fulfill certain standards:
\begin{description}
  \item[Exit status:] The exit status returned must be zero if all went well and
    non-zero if any error occured.
  \item[Idempotency:] The scripts must not throw an error or harm the users
    system if called repeatedly. 
  \item[Dependencies:] All programs used in the scripts (as well as the
    interpreter for the script) must be listed in the Pre-Depends field of the
    control file (see chapter \ref{control}, p. \pageref{control}).
  \item[Language:] Any script-language \emph{works}, but it is recommended to
    rely on a language where the interpreter is an essential package and thus
    installed on every debian-based system to keep dependencies to a minimum.
\end{description}
The scripts are called with different paramaters, depending if the current
installation process is a new installation, an upgrade, a removal, etc..
Since there is a plethora of possible parameters these scripts can be called
with, and most of them were never used for the \tunix project, please refer to
\cite{debian policy manual} for a complete documentation.

\subsubsection{Conffiles}
If your package includes configuration files that might be changed by the user,
you should list them, one per line, in \file{DEBIAN/conffiles}. If a file
is listed there, dpkg will ask the user what to do if a conffile was changed and
a new version is available. 

%%%%%%%%%%%%%%%%%%%%%%%%%%%%%%%%%%%%%%%%%%%%
%%% Package building & quality assurance %%%
%%%%%%%%%%%%%%%%%%%%%%%%%%%%%%%%%%%%%%%%%%%%
\subsection{Package building \& quality assurance}\label{subsection:package %
building} Once all the files are in place the package building can begin. Code
example \ref{code:building filelist} (p. \pageref{code:building filelist}) shows
all necessary files.  This package also includes a postinst and a prerm script
as well as a binary called "something" that will be installed into
\file{/usr/bin/}.
\begin{lstlisting}[label=code:building filelist,caption=A directory ready
to be made into a .deb package, float=ht]
$\sim$/work $\$$ find -type f new/
new/DEBIAN/control
new/DEBIAN/postinst
new/DEBIAN/prerm
new/usr/share/doc/new/changelog
new/usr/share/doc/new/changelog.Debian
new/usr/share/doc/new/copyright
new/usr/bin/something
\end{lstlisting}

Once you know everything is there, some preparations have to be done:
\begin{itemize}
  \item compress files found in \file{usr/share/doc}
  \item recreate the md5sums
\end{itemize}
Code example \ref{code:building preparation} (p. \pageref{code:building
preparation}) shows the commands necessary to make the directory ready:
\begin{lstlisting}[label=code:building preparation,caption=prepare a directory for
package creation, float=ht]
# compress changelog and changelog.Debian
$\sim$/work $\$$ cd new/usr/share/doc/new
$\sim$/work/new/usr/share/doc/new $\$$ ls
changelog changelog.Debian copyright
$\sim$/work/new/usr/share/doc/new $\$$ gzip --best changelog
$\sim$/work/new/usr/share/doc/new $\$$ gzip --best changelog.Debian
$\sim$/work/new/usr/share/doc/new $\$$ ls
changelog.gz changelog.Debian.gz copyright

# recreate md5sums
$\sim$/work/new/usr/share/doc/new $\$$ cd $\sim$/work/new
$\sim$/work/new $\$$ rm -f DEBIAN/md5sums
$\sim$/work/new $\$$ find * -type f | grep --invert-match \ 
	"^DEBIAN[$\$$/]" | xargs md5sum > DEBIAN/md5sums
$\sim$/work/new/new $\$$ cd $\sim$/work
\end{lstlisting}

Once all the preparations are done, building the package takes only the single
command found in line 2 of \lstlistingname{} \ref{code:building} (p.
\pageref{code:building}). After successfully building  the package, you should
test its adherence to debian standards using the tool \cmd{lintian}. If it
produces a warning you may ignore it if you have a good reason, but errors must
be fixed. If the error-code produced by lintian is not clear, you can get
further information with \cmd{lintian-info -t \$error\_code}, where 
\$error\_code is the code produced by \cmd{lintian}.
\begin{lstlisting}[label=code:building,caption=create a package, float=ht]
# actually build the package:
$\sim$/work $\$$ dpkg-deb --build new/ new_1.0.0-1_i386.deb
$\sim$/work $\$$ lintian new_1.0.0-1_i386.deb
\end{lstlisting}


%%%%%%%%%%%%%%%%%%%%%%%%
%%% Apt repositories %%%
%%%%%%%%%%%%%%%%%%%%%%%%
\section{Apt repositories}\label{chapter:apt repositories}
The .deb packages created in the process described above are nice, but in order
to be able to install them via \cmd{apt}, they have to be put into a
(possibly remote) \emph{apt repository}. A repository is a special directory hierarchy
with a few special files containing (meta)data. A repository might have packages
for multiple distributions, multiple components and  multiple architectures.
Complete documentation (especially on signing packages) is hard to come by, the 
official (but incomplete) tutorial is the Debian Repository
HOWTO\cite{debian repository howto}.

While creating .deb source packages wasn't described in the previous chapter,
creating an apt repository for source packages is described as well, since it
almost comes for free when creating a repository for binary packages.

Whenever paths are given, they are relative to the repository root and \$dist,
\$comp and \$arch denote the respective levels in the repository layout as
outlined by subsection \ref{directory structure} (p. \pageref{directory
structure}).

The commands given here are 100\% scriptable. The whole maintenance process is
taken care of by the script \cmd{repo-maint.sh} found in the scripts-directory
of the \tunix homepage.

%%%%%%%%%%%%%%%%%%%%%%%%%%%
%%% Directory structure %%%
%%%%%%%%%%%%%%%%%%%%%%%%%%%
\subsection{Directory structure}\label{directory structure}
The basic directory structure of an apt repository is rather simple. 
The \file{dists/} directory contains directories for
each distribution (\$dist) this repository has packages for, each in turn
includes one or more directories representing components (\$comp) that may have
source packages and binary packages (\$arch) for any number of distributions. A
real-world example can be found in \lstlistingname{} \ref{code:apt-repo directories}
(p. \pageref{code:apt-repo directories}).

\$dist may be any distribution you release packages for, \$comp is used to
divide the distribution into independent components. The official Ubuntu
repositories use "main", "universe" and "multiverse", Debian uses "free",
"non-free" and "contrib". Within every component there can be a \file{source/}
directory and/or directories of the form \file{binary-} with the CPU
architecture as a suffix. Since \tunix has only few packages, there is only one
distribution ("tunix") and the component "main", with packages for amd64 and
i386.

You also need a directory for all your .deb packages. It can have any name and
can be located anywhere. In the following chapters the directory \file{all} in
\file{dists/\$dist/\$comp} is used.

\begin{lstlisting}[label=code:apt-repo directories,caption=Basic structure of an
apt repository, float=ht]
(repository root)
|
+- dists
   |
   +- tunix
     |
     +- main
	|
	|-all
	|-binary-i386
	|-binary-amd64
	|-source
\end{lstlisting}

%%%%%%%%%%%%%%%%
%%% Metadata %%%
%%%%%%%%%%%%%%%%
\subsection{Metadata}
\subsubsection{Override file}
The override file is optional but recommended. It contains information the
package maintainer wouldn't know but the distributor would. The
syntax is easy: It lists, one per line, all packages together with its category
and priority. The category might be anything the distributor thinks of, the
priority might be one of "extra", "optional", "standard", "important" or
"required". A working example for three
packages can be found in \lstlistingname{} \ref{code:override} (p.
\pageref{code:override}).
\begin{lstlisting}[label=code:override,caption=override file, float=ht]
dia-deskfile standard gnome
eprog-io optional misc
tunix required base
\end{lstlisting}

The filename and location of the file is actually up to the user, though the
script \cmd{repo-maint.sh} assumes the file is in \file{dists/\$dist/\$comp/}
and is called \file{override.\$dist.\$comp}.

\subsubsection{Sources/Packages file}
The most important file of the apt repository is the list of packages available
in the repository. For source repositories the file is named \file{Source}, for
binary repositories the file is named \file{Packages} and must be present in
every \$arch directory. The file is a detailed list of all packages and their
meta-information (composed of the packages control-file and the repositories
override-file) and should be available as a plain-text file and in gzip-
and bzip2-compressed form.

These files are created using \cmd{dpkg-scansources} (for \file{Sources}) 
respectively \cmd{dpkg-scanpackages} (for \file{Packages}). You can find a
demonstration (that would work in the \tunix apt repository) of the necessary
commands in \lstlistingname{} \ref{code:packages} (p. \pageref{code:packages}).

\begin{lstlisting}[label=code:packages,caption=Creating Sources and Packages
files,float=ht]
# create Sources file
/apt/dists/tunix/main $\$$ dpkg-scansources all \
	override.tunix.main > source/Sources
/apt/dists/tunix/main $\$$ cat source/Sources | gzip -9c > \
	source/Sources.gz
/apt/dists/tunix/main $\$$ cat source/Sources | bzip2 -z9 > \
	source/Sources.bz2

# create packages file for binary-i386
/apt/dists/tunix/main $\$$ dpkg-scanpackages -ai386 all \
	override.tunix.main > binary-i386/Packages
/apt/dists/tunix/main $\$$ cat binary-386/Packages | \
	gzip -9c > binary-i386/Packages.gz
/apt/dists/tunix/main $\$$ cat binary-i386/Packages | \
	bzip2 -z9 > binary-i386/Packages.bz2
\end{lstlisting}

\subsubsection{Release file}
There are actually two different kinds of \file{Release} files. One is located in
\file{dists/\$dist/\$comp/\$arch/} and contains checksums for the
\file{Packages} respectivly \file{Sources} file in compressed and uncompressed
form. The other one is located in \file{dists/\$dist} and contains checksums for all
\file{Sources}, \file{Packages} and \file{Releases} in its subdirectories. Both
have a common static head. 

The demonstration found in \lstlistingname{} \ref{code:release file} (p.
\pageref{code:release file}) creates a \file{Release} file (the syntax is the
same for all files). The \cmd{grep} is
necessary so the file doesn't include a (wrong) checksum of itself. It is
assumed that the common static head is found in \file{dists/$\$$dist/Release.head}.
\begin{lstlisting}[label=code:release file, caption=Creating a Release file,
float=ht]
/apt/dists/tunix $\$$ export release="main/source/Release"
/apt/dists/tunix $\$$ cat Release.head > $\$$release
/apt/dists/tunix $\$$ echo "Architecture: source" >> $\$$release
/apt/dists/tunix $\$$ echo "Component: main" >> $\$$release
/apt/dists/tunix $\$$ apt-ftparchive release main/source/ \
	| grep --invert-match \
		"^ [0-9a-z]* *[0-9]* Release$\$$" \
	>> $\$$release
\end{lstlisting}

\subsubsection{Signing the repository}
For added security the \file{Release} files have to be signed using GnuPG. If
the files are not singed, the user will get a warning when installing packages
from the repository. The demonstration in \lstlistingname{} \ref{code:gpg
signing} (p. \pageref{code:gpg signing}) assumes that the public and private key
is to the signer is available, the key used for signing is foo@bar.com and the
password for that key is "gpgpass".
\begin{lstlisting}[label=code:gpg signing, caption=Signing a Release file,
float=ht]
/apt/dists/tunix $\$$ echo "gpgpass" | \
	gpg --passphrase-fd 0 -a -b -s -q \
	-u "foo@bar.com" --batch -o Release.gpg Release
\end{lstlisting}
