% ideas for tunix.tex

%%%%%%%%%%%%%%%%%%%%%%%%
% Some ideas for TUn!x %
%%%%%%%%%%%%%%%%%%%%%%%%
\chapter{Some ideas for \tunix}
Several different ideas come up when thinking of a live-CD specialized for a
specific environment. This chapter lists what we could think of, but only some
of the ideas given here where implemented. Althought theoretically you can do
almost anything with an Ubuntu-based live-CD, what is possible is limited by
several constraints:
\begin{description}
 \item[manpower] is limited. While Ubuntu has dozens of skilled technicians and
   a large community supporting it, \tunix has two persons working on it over
   the summer holydays.
 \item[available space:] The Ubuntu live-CD as it is leaves only a few kilobytes
   free for modifications. Very little space is required for simple
   configuration changes, but if any new software is to be installed, other
   software has to be removed. Even after the removal of windows software and the
   example-content package, only about 30MB were available.
 \item[maintenance] after the summer holydays is voluntary and cannot be
   guaranteed. If you consider adding modified packages of existing software,
   you have to keep them up-to-date until at least the next version of \tunix is
   released.
\end{description}
Several ideas are presented on the next pages and only some of them where
implemented. If they were implemented, you can find documentation on it in the
following chapters, if not, a reason is given.
\begin{description}
  \item[apt-repository] The core of any Debian-based distribution is Apt \& DPKG,
    a central package management-system. Using an apt-repository has the
    advantage that updates to whatever makes \tunix differ from Ubuntu can be
    made in a central and automated matter, fully integrated into the
    distribution.
  \item[community] A community providing tipps, tricks and most importantly
    feedback to the developers is a great asset to any software. Computer
    Science students have a large forum\footnote{http://www.informatik-forum.at}
    where a \tunix subforum is available. The \tunix
    homepage\footnote{http://www.informatik.tuwien.ac.at/tunix} is a Wiki that
    allows collaborative writing of documentation. 

    Unfortunatly, the \tunix project doesn't really have the resources to
    "maintain" a community. The developers implement \tunix as a summer project,
    but users use it at the start of the semester. 
  \item[applications] All students use some software that doesn't come
    preinstalled with Ubuntu. Having this software preinstalled is the
    primary goal for \tunix.
  \item[offline documentation] Some stuff relevant for students could be
    directly on the CD. This could include course descriptions, building
    plans\todo{aeh, stimmt das so? "Gebaeudeplaene"} and important legal
    documents.
  \item[localized applications] Software could be preconfigured for typical
    use in a TU-Environment. The webbrowser could have localized bookmarks,
    the mail client could be predconfigured for the students email address,
    the VPN-client could have configuration files for the TU wireless
    network, and so on. Of the aforementioned, only the configuration of
    the mail client wasn't implemented.
  \item[missing applications] Some more applicitions are simply missing in
    Ubuntu. Primarily, this includes Latex, subversion, ntfs-support,
    various compilers, etc. \tunix has sysv-rc-conf (a small ncurses-based
    programm that eases startup-configuration) and gajim (a good Jabber
    instant messaging client) preinstalled. Other programs weren't
    installed due to space-constraints (Latex) or pre-alpha status of the
    software (NTFS support).
  \item[individial look]
\end{description}

\todo{write this chapter}
