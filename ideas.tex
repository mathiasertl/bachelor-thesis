% ideas for tunix.tex

\chapter{Some ideas for \tunix}
Several different ideas come up when thinking of a live-CD specialized for a
specific environment. This chapter lists what we could think of, but only some
of the ideas given here where implemented. Althought theoretically you can do
almost anything with an Ubuntu-based live-CD, what is possible is limited by
several constraints:
\begin{description}
 \item[manpower] is limited. While Ubuntu has dozens of skilled technicians and
   a large community supporting it, \tunix has two persons working on it over
   the summer holydays.
 \item[available space:] The Ubuntu live-CD as it is leaves only a few kilobytes
   free for modifications. Very little space is required for simple
   configuration changes, but if any new software is to be installed, other
   software has to be removed. Even after the removal of windows software and the
   example-content package, only about 30MB were available.
 \item[maintenance] after the summer holydays is voluntary and cannot be
   guaranteed. If you consider adding modified packages of existing software,
   you have to keep them up-to-date until at least the next version of \tunix is
   released.
\end{description}
Several ideas are presented on the next pages and only some of them where
implemented. If they were implemented, you can find documentation on it in the
following chapters, if not, a reason is given.
\begin{description}
  \item[apt-repository] The core of any Debian-based distribution is a central
    package management-system. 
  \item[community] (wiki, forum)
  \item[applications] java, postgresql, dia
  \item[offline documentation] studienplan, lageplaene
  \item[localized applications] firefox bookmars, mail-client, vpnc
  \item[missing applications] sysv-rc-conf, latex, ntfs-support
  \item[individial look]
\end{description}

\todo{write this chapter}
