% CD remastering for tunix.tex

%%%%%%%%%%%%%%%%%%
% CD Remastering %
%%%%%%%%%%%%%%%%%%
\chapter{CD remastering}\label{chapter:CD remastering}
CD remastering describes the process of disassembling an existing live-CD,
modifying it to your needs and reassembling it again. This chapter describes
what is required for the process, the contents of the live-CD, how to unpack it,
modify with it, and put it back together again.

%%%%%%%%%%%%%%%%
% Requirements %
%%%%%%%%%%%%%%%%
\section{Requirements}
The process has rather high requirements:
\begin{description}
 \item[Hardware:] $\qquad$
   \begin{itemize}
    \item A fast CPU and lots of RAM is required. Building a SquashFS with a
      Pentium-M 1.5 GHz and 512MB RAM takes about half an hour.
    \item You need even more RAM when testing your live-CD on a virtual
      machine. 2GB RAM is recommended.
    \item The harddisc needs to be large (overall disc-requirement is at least
      5GB at any time) and fast (Building an ISO requires little CPU-power and
      is is mainly dependent on HDD-speed).
   \end{itemize}
 \item[Software:] $\qquad$
   \begin{itemize}
     \item You have to have Linux with a kernel with SquashFS support and the
       appropriate filesystem-tools. Under Ubuntu, everything can be installed
       with the packages \pkg{squashfs-tools} and \pkg{squashfs-source}.
     \item Technically not absolutely necessary, but you \emph{will} save a lot
       of time and many burned CDs if you have a virtual machine on hands.
       Qemu\footnote{http://fabrice.bellard.free.fr/qemu/} or the non-free
       VMware\footnote{http://www.vmware.com/} are just some alternatives.
   \end{itemize}
 \item[Knowledge:] $\qquad$
   \begin{itemize}
     \item Very good shell-scripting knowledge. Recommended reading is available at
     	\cite{bash-intro} and \cite{bash-advanced}.
     \item The ability to live with very bad documentation. Some of the tools
       described and used here are \emph{poorly} documented
   \end{itemize}
\end{description}

%%%%%%%%%%%%%%%%%%%%
% Live-CD contents %
%%%%%%%%%%%%%%%%%%%%
\section{Live-CD contents}
The \tunix live-CD contains mainly the following files and directories relevant
to the process of CD remastering:
\begin{description}
  \item[\file{isolinux/}] contains the bootloader
  \item[\file{casper/}] contains all following files:
  \begin{description}
    \item[\file{vmlinuz}] and \textbf{\file{initrd.gz}} is the kernel and RAM-disc
      the live-CD boots from.
    \item[\file{filesystem.squashfs}] is the root filesystem used by the kernel
      mentioned above.
    \item[\file{filesystem.manifest}] contains a list of packages installed on the
      root filesystem mentioned above.
    \item[\file{filesystem.manifest-desktop}] contains a list of packages that the
      installer will install on the harddisc. For practical use, this list is a
      subset of the list in filesystem.manifest.
  \end{description}
\end{description}

%%%%%%%%%%%%%%%%%%%%%%%%%%%
% Overview of the process %
%%%%%%%%%%%%%%%%%%%%%%%%%%%
\section{Overview of the process}
The process can be easily summarized by the following steps:
\begin{itemize}
  \item Dump the ISO to the harddisc
  \item Modify the contents of the ISO filesystem (*)
  \item Unpack the SquashFS file (requires up to 2GB space!)
  \item Chroot into the SquashFS (this is the environment the live-CD will boot
    into).
  \item Modify the contents of the SquashFS filesystem (*)
  \item Rebuild the manifest-files
  \item Rebuild SqashFS
  \item Rebuild ISO
  \item Test ISO in a virtual machine
\end{itemize}
The two steps marked with a "(*)" represent the steps that actually modify the
CD. You don't \emph{have} to modify either the ISO or the SquashFS, but doing
neither seems somewhat pointless. The whole process is, by the way, very
scriptable.

The whole process is described here using example shell commands. But the entire
process was automated into shell scripts. The scripts can be found in the
scripts-directory of the \tunix-homepage, here is an overview of what they do:
\begin{description}
  \item[\file{make-env.sh}] dumps the ISO, unpacks the SquashFS, creates all
    other scripts mentioned here as well as a file \file{.settings} that is
    included by these scripts so they can share some common settings.
  \item[\file{chroot.sh}] chroots into the unpacked SquashFS, does some
    cleanup after the user exits the chrooted shell and finally recreates the
    manifest-files of the SquashFS.
  \item[\file{mksquashfs.sh}] rebuilds the SquashFS
  \item[\file{mkisofs.sh}] rebuilds md5 checksums and creates an ISO that is
    ready to boot. The ISO is
    named \file{tunix\_\$date-r\$N.iso} where \$date is the output of the
    command \cmd{date ---iso-8601} and \$N is an auto-incremented
    revision-number.
\end{description}

%%%%%%%%%%%%%%%%%%%%%%%%%%%%%%%%%%%%%%%
% Detailed description of the process %
%%%%%%%%%%%%%%%%%%%%%%%%%%%%%%%%%%%%%%%
\section{Detailed description of the process}
This section describes the CD remastering process in more detail. The code
snippets provided include a prompt starting with the working directory followed
by a '\$' (normal user) or a '\#' (root user) depending on the rights required.
A line starting with a '\#' is considered a comment\footnote{Note that bash
usually considers everything starting from the first '\#' a comment. The syntax
that a line starting with a '\#' is a comment resembles the syntax of a shell
script and is used to avoid confusion with a root shell-prompt.}. The
base-directory is always the directory \file{$\sim$/work}.

%%%%%%%%%%%%%%%%
% Preparations %
%%%%%%%%%%%%%%%%
\subsection{Preparations}
This section assumes you have already have the ISO image you want to modify. The
code-snippet found in \lstlistingname{} \ref{code:prepare} (p. \pageref{code:prepare})
mounts the ISO image \file{original.iso} as a loop-device, copies its
contents to the directory \file{live-cd/} and, in the second part, unpacks
the SquashFS to the directory \file{squashfs/}. 
\begin{lstlisting}[caption={Unpacking the ISO and SquashFS filesystems},float=ht,label=code:prepare]
# unpack ISO:
$\sim$/work $\$$ mkdir tmp/
$\sim$/work $\$$ mount -t iso9660 -o loop ubuntu.iso tmp/
$\sim$/work $\$$ mkdir live-cd/
$\sim$/work $\$$ cp -a tmp/. live-cd/
$\sim$/work $\$$ chmod -R u+w live-cd/
$\sim$/work $\$$ umount tmp/

# unpack SquashFS:
$\sim$/work $\$$ mount -t squashfs -o loop,ro \
	live-cd/casper/filesystem.squashfs tmp/
$\sim$/work $\$$ cp -a tmp/. squashfs/
$\sim$/work $\$$ umount tmp/
$\sim$/work $\$$ rmdir tmp/
\end{lstlisting}

%%%%%%%%%%%%%%%%%%%%%%%%%%%%%%%%%%%%%%%%%%
% Chrooting into the live-CD environment %
%%%%%%%%%%%%%%%%%%%%%%%%%%%%%%%%%%%%%%%%%%
\subsection{Chrooting into the live-CD environment}
If you want to modify the live-CD and what will be installed on the
computer if the user chooses to install it, you will want to chroot, or
\emph{change root}, into the unpacked SquashFS filesystem that was unpacked in
the previous chapter. Chrooting means executing a command (and may be an
interactive shell) with a different root-directory than the "real" one. This
allows you to use all the tools available on the live-CD to modify it to your
needs. The necessary commands are described in \lstlistingname{} \ref{%
code:chroot} (p. \pageref{code:chroot}). The last line is there to emphasize
that 
\begin{itemize}
\item all paths (relative or absolute), including targets of symlinks, now refer
to destinations inside the new root directory.
\item you effectively have the rights of the root user in that environment,
since all files (including programs with the suid-bit set) are owned by you.
\end{itemize}

A word of warning: The tools used in the chrooted environment will behave
like they would on the live-CD or the installed system\ldots{} almost. You have
to keep in mind that even if you chrooted into the live-CD, you never
\emph{booted} from it. All the \emph{regular files} might be the same as on the
live-CD, but the filesystems in \file{/dev}, \file{/proc} and \file{/sys} are
dynamically generated at every boot and differ depending on the hardware of the
computer. The behaviour of programs is also influenced by the environment, e.g.
environment variables, kernel-version, etc., and only some of that can be
reliably emulated. For example, it would be problematic to install packages that
install new kernel modules, if the kernel-version on the live-cd is not
\emph{exactly} that of the host, since the kernel-version reportet by
\cmd{uname -a} is still that of the host and not that of the live-CD.
The code found in \lstlistingname{} \ref{code:chroot} (p. 
\pageref{code:chroot}) also shows a few preparations necessary to get all the
tools in the chrooted environment to work.\\

\begin{lstlisting}[caption={Chrooting into the SquashFS filesystem},float=ht, %
label=code:chroot]
# preparations:
$\sim$/work $\$$ cp /etc/resolv.conf squashfs/etc/
$\sim$/work $\$$ mount -t proc -o bind /proc squashfs/proc
$\sim$/work $\$$ mount -t udev -o bind /dev squashfs/dev

# actual chroot:
$\sim$/work $\$$ chroot new/ /bin/bash
/ # 
\end{lstlisting}

After modifying the SquashFS to your needs, you should do some
cleaning up\footnote{Completely cleaning up is only necessary if you want to know
the exact size of the final ISO or want to create a final release}. The
command-sequence shown in \lstlistingname{} \ref{code:chroot-cleanup} (p.
\pageref{code:chroot-cleanup}) starts in the chrooted environment and exits into
the real environment on line 3 with the command \cmd{exit}.
\begin{lstlisting}[caption={Cleanup the SquashFS},float=ht,%
	label={code:chroot-cleanup}]
/ # apt-get clean
/ # rm /etc/resolv.conf
/ # exit
$\sim$/work $\$$ umount squashfs/proc squashfs/dev
\end{lstlisting}
%\todo{}

%%%%%%%%%%%%%%%%%%%%%%%
% Rebuild the live-CD %
%%%%%%%%%%%%%%%%%%%%%%%
\subsection{Rebuild the live-CD}
All that is left to do now is to rebuild the SquashFS and the ISO filesystems so
you can test them on your virtual machine. Before you can rebuild the ISO
filesystem, you have to update the manifest-files (of the SquashFS) and
md5-checksums of all files on the ISO. The process is described in \lstlistingname{}
\ref{code:rebuild} (p. \pageref{code:rebuild}). Note that rebuilding the SquashFS
filesystem takes especially long - consider getting a cup of coffee.

\begin{lstlisting}[caption={Rebuiling the live-CD},float=ht,label={code:rebuild}]
# rebuild the manifest files:
$\sim$/work $\$$ chroot squashfs/ dpkg-query -W --showformat=\
	'${Package} ${Version}\n' > \
	"live-cd/casper/filesystem.manifest"
$\sim$/work $\$$ cat > /tmp/tmp.control <<FOO
/casper/d
/libdebian-installer4/d
/os-prober/d
/ubiquity/d
/ubuntu-live/d
/user-setup/d
FOO 
$\sim$/work $\$$ sed -f /tmp/$$.control < \
	"live-cd/casper/filesystem.manifest" > \
	"live-cd/casper/filesystem.manifest-desktop"
$\sim$/work $\$$ rm /tmp/tmp.control

# recreate squashfs
$\sim$/work $\$$ rm live-cd/casper/filesystem.squashfs
$\sim$/work $\$$ cd squashfs/
$\sim$/work/squashfs $\$$ mksquashfs . \
	../live-cd/casper/filesystem.squashfs

# recreate md5 checksums:
$\sim$/work $\$$ cd live-cd
$\sim$/work/live-cd $\$$ find . -type f -print0 | \
	xargs -0 md5sum | tee md5sum.txt

# create ISO:
$\sim$/work/live-cd $\$$ mkisofs -o "neues.iso" -b \
	isolinux/isolinux.bin -c isolinux/boot.cat \
	-no-emul-boot -boot-load-size 4 \
	-boot-info-table -r -V "My live-CD" \
	-cache-inodes -J -l live-cd
\end{lstlisting}
