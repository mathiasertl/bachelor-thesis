% live-modifications for tunix.tex

\chapter{Modifications to the live-CD}\label{chapter:modifications}

\section{APT}
The \tunix{} apt repository had to be integrated into the \tunix{} live-CD so that
the users will actually use the new repository. This requires only two steps:
\begin{enumerate}
  \item sources.list: \file{/etc/apt/sources.list} has to be modified to include
    the repository. In the \tunix{} distribution the file includes the
    line\footnote{the domain apt.fsinf.at could have been used instead, but
    wasn't available at the time of shipping.}
    \begin{quote}deb http://pluto.htu.tuwien.ac.at/apt main\end{quote}
  \item gpg key: the public gpg key used for signing the repository has to be
    known to apt to accept it. This can be done with the simple command
    \cmd{apt-key add pubkey.gpg} where \file{pubkey.gpg} is the public key.
\end{enumerate}

\section{Mozilla Firefox}
Mozilla Firefox, the standard webbrowser of Ubuntu/\tunix, was modified so that
it includes bookmarks to the websites that are most important for any computer
science student. Changing the bookmarks is very simple if you use the
following trick: Boot the live-CD, start Firefox, adjust the bookmarks to your
liking and copy the bookmarks file (search for \file{bookmars.html} in
\file{$\sim$/.mozilla/firefox/}) to \file{/etc/firefox/profile} in
the SquashFS.

\section{USplash}
USplash is used to view a nice image instead of the hundreds of lines of output
from the kernel while booting and shutting down \tunix. Some unofficial
documentation is available at \cite{USplashCustomizationHowto}, but USplash is
otherwise poorly documented. 

The first step is to create a PNG image file that you want displayed when
booting/shutting down. It has to be an indexed PNG with 16 colors\footnote{note
that this seems to violate the PNG standard, but GIMP is able to create such
files} with 640x400 pixels (not, as the documentation states, 640x4\textbf{8}0).
The colors used in the index are also the colors used in the dynamic elements
(progress bar, etc.) of the splashscreen. The following table describes what
positions of the index are used for which elements:\\

\begin{tabular}[ht]{|c|p{11cm}|}
\hline
\textbf{index} & \textbf{what its used for} \\ \hline
0 & background of the text output \\ \hline
1 & progress bar \\ \hline
2 & status message of init-scripts ("[ok]" most of the time) \\ \hline
4 & background of the progress bar \\ \hline
8 & text on the left side of the output used by the init-scripts to say what
they are doing \\ \hline
13 & error output in the text-field (to try this out: don't create any md5sums)
\\ \hline
\end{tabular}\\\\

Next, the PNG has to be converted into a shared object file (suffix .so). This
is achieved by converting the PNG into a bogl file (bogle is a subset of the C
programming language) using \cmd{pngtobogl} (available in the package
libbogl-dev) and then compiling it into a shared object file using \cmd{gcc}.
The process is described in \lstlistingname{} \ref{code:png to so} (p.
\pageref{code:png to so}).
\begin{lstlisting}[float=ht,label=code:png to so,caption=Converting a PNG to a
shared object]
$\sim$/tunix $\$$ pngtobogl new-splash.png > new-splash.c 
$\sim$/tunix $\$$ gcc -Os -g -I/usr/include/bogl -fPIC -c \
	new-splash.c -o new-splash.o
$\sim$/tunix $\$$ gcc -shared -Wl,-soname,usplash-artwork.so \
	new-splash.o -o new-splash.so
$\sim$/tunix $\$$ strip --strip-unneeded new-splash.so
$\sim$/tunix $\$$ rm new-splash.[co]
\end{lstlisting}

Sadly, the process is not always that simple. At least in the version available
08/2006, \cmd{pngtobogl} created a buggy file in which all the first (0) and second
(1) entries in the palette were flipped, causing the picture to be partly inverted.
This can be fixed with a few simple \cmd{sed} scripts, but is a source of
frustration.

To get the live-CD to use the new image it is necessary to unpack the inital
RAM-disc (\file{initrd.gz}) of the kernel found on the live-CD, replace the .so
file and repack the initial RAM-disc. The procedure is described in
\lstlistingname{} \ref{code:initrd live} (p. \pageref{code:initrd live}). 
\begin{lstlisting}[float=ht,label=code:initrd live,caption=Replacing the
splashscreen on the live-CD]
$\sim$/tunix $\$$ mkdir tmp
$\sim$/tunix $\$$ cp live-cd/casper/initrd.gz tmp/
$\sim$/tunix $\$$ cd tmp/

# unpack the RAM disc:
$\sim$/tunix/tmp $\$$ cat initrd.gz | gzip -d | cpio -i
$\sim$/tunix/tmp $\$$ rm initrd.gz 

# replace the splash-image
$\sim$/tunix/tmp $\$$ cp ../new-splash.so \ 
	usr/lib/usplash/usplash-artwork.so

# rebuild the ram disc:
$\sim$/tunix/tmp $\$$ find | cpio -H newc -o \
	| gzip > ../live-cd/casper/initrd.gz
\end{lstlisting}

To make the installed system use the splashscreen, you have to use a different
approach that makes use of debians \cmd{update-alternatives} system. You have to
\cmd{chroot} into the unpacked SquashFS, for which we use the script chroot.sh
that does the chrooting process as defined in chapter \ref{chapter:CD
remastering} (p. \pageref{chapter:CD remastering}). The process is described in
\lstlistingname{} \ref{code:initrd install} (p. \pageref{code:initrd install}).
Line 10 is particulary interesting: most tutorials suggest using
\cmd{dpkg-reconfigure linux-image-\$(uname -r)} but this won't work since
\cmd{uname -r} will return the version of the kernel installed on \emph{your}
system, not the one in the SquashFS. To find out which kernel-version is
installed in the SquashFS you can use the command \cmd{dpkg-query -W |
grep linux-image}.
\begin{lstlisting}[float=ht,label=code:initrd install,caption=Replacing the
splashscreen on the installed system]
$\sim$/tunix $\$$ cp new-splash.so \
	squashfs/usr/lib/usplash/usplash-tunix.so
$\sim$/tunix $\$$ ./chroot.sh
$\sim$ # cd /usr/lib/usplash
/usr/lib/usplash # update-alternatives --install \
	/usr/lib/usplash/usplash-artwork.so \
	usplash-artwork.so \
	/usr/lib/usplash/usplash-tunix.so 55
$\sim$ # update-alternatives --config usplash-artwork.so
$\sim$ # dpkg-reconfigure linux-image-2.6.15-23-386
\end{lstlisting}

\section{VPNC}
Two extra configuration files for VPNC were added to the live-CD, one for use
in the WLAN of the Technical University of Vienna (\file{/etc/vpnc.wlan.conf})
and one if you want to use VPN from outside of the university network
(\file{/etc/vpnc.conf}).
\begin{lstlisting}[float=ht,label=code:vpnc.conf,caption=VPNC config-file for
use from outside the University]
# this is the vpnc config file is used if you connect from
# outside of the # Technical University, i.e. to encrypt
# your connection over an open WLAN.

Interface name vpntun0
IPSec gateway terminator.tuwien.ac.at
IPSec ID vpncclient
IPSec secret vpnc2tu
# enter your "Matrikelnummer" here and uncomment the
# following line:
#Xauth username e0123456@vpn.tuwien.ac.at
# It is also to put your password here, but be aware of the
# security impliciations of having your password in
# plaintext in a file.
#Xauth password blabla
Debug 1
\end{lstlisting}

\begin{lstlisting}[float=ht,label=code:vpnc.wlan.conf,caption=VPNC config-file for
use in the TU of Vienna WLAN]
# This config file is used if you want to encrypt your
# connection from inside the Technical University (wlan
# or ethernet).

Interface name vpntun0
IPSec gateway terminator.demo.tuwien.ac.at
IPSec ID vpncmobil
IPSec secret mobilviavpnc
# enter your "Matrikelnummer" here and uncomment the
# following line:
#Xauth username e0123456@mobil.tuwien.ac.at
Debug 1
\end{lstlisting}

\section{GNU Screen}
In \tunix, GNU Screen uses UTF8 per default and displays a fancy "window bar" at
the bottom of the screen to improve navigation. This is achieved by an
additional file \file{/etc/skel/.screenrc}. The contents of the \cmd{/etc/skel}
directory are usually copied into newly created home-directories.

Please note that the last line is split into two lines in \lstlistingname{}
\ref{code:screenrc} for typesetting reasons, but this has to be one line in an
actual file.
\begin{lstlisting}[float=ht,label=code:screenrc,caption=Additional configuration
for GNU Screen]
# TUn!x: make screen use utf8:
defutf8 on
# TUn!x: adds the "window-bar" on the bottom:
hardstatus alwayslastline \
"%{= BY}%-Lw%{= YB}%50>%n%f* %t%{= BY}%+Lw%<%= %m/%d %c"
\end{lstlisting}

%\section{Gnome/GTK+/GDM}
%\todo{describe the changes}
